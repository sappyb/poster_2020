%%%%%%%%%%%%%%%%%%%%%%%%%%%%%%%%%%%%%%%%%
% Jacobs Landscape Poster
% LaTeX Template
% Version 1.1 (14/06/14)
%
% Created by:
% Computational Physics and Biophysics Group, Jacobs University
% https://teamwork.jacobs-university.de:8443/confluence/display/CoPandBiG/LaTeX+Poster
% 
% Further modified by:
% Nathaniel Johnston (nathaniel@njohnston.ca)
%
% This template has been downloaded from:
% http://www.LaTeXTemplates.com
%
% License:
% CC BY-NC-SA 3.0 (http://creativecommons.org/licenses/by-nc-sa/3.0/)
%
%%%%%%%%%%%%%%%%%%%%%%%%%%%%%%%%%%%%%%%%%

%----------------------------------------------------------------------------------------
%	PACKAGES AND OTHER DOCUMENT CONFIGURATIONS
%----------------------------------------------------------------------------------------

\documentclass[final]{beamer}

\usepackage[scale=1.24]{beamerposter} % Use the beamerposter package for laying out the poster

\usetheme{confposter} % Use the confposter theme supplied with this template

\setbeamercolor{block title}{fg=ngreen,bg=white} % Colors of the block titles
\setbeamercolor{block body}{fg=black,bg=white} % Colors of the body of blocks
\setbeamercolor{block alerted title}{fg=white,bg=dblue!70} % Colors of the highlighted block titles
\setbeamercolor{block alerted body}{fg=black,bg=dblue!10} % Colors of the body of highlighted blocks
% Many more colors are available for use in beamerthemeconfposter.sty

%-----------------------------------------------------------
% Define the column widths and overall poster size
% To set effective sepwid, onecolwid and twocolwid values, first choose how many columns you want and how much separation you want between columns
% In this template, the separation width chosen is 0.024 of the paper width and a 4-column layout
% onecolwid should therefore be (1-(# of columns+1)*sepwid)/# of columns e.g. (1-(4+1)*0.024)/4 = 0.22
% Set twocolwid to be (2*onecolwid)+sepwid = 0.464
% Set threecolwid to be (3*onecolwid)+2*sepwid = 0.708

\newlength{\sepwid}
\newlength{\onecolwid}
\newlength{\twocolwid}
\newlength{\threecolwid}
\setlength{\paperwidth}{48in} % A0 width: 46.8in
\setlength{\paperheight}{36in} % A0 height: 33.1in
\setlength{\sepwid}{0.024\paperwidth} % Separation width (white space) between columns
\setlength{\onecolwid}{0.22\paperwidth} % Width of one column
\setlength{\twocolwid}{0.464\paperwidth} % Width of two columns
\setlength{\threecolwid}{0.708\paperwidth} % Width of three columns
\setlength{\topmargin}{-0.5in} % Reduce the top margin size


%-----------------------------------------------------------
\usepackage[most]{tcolorbox}
\usepackage{graphicx}  % Required for including images
\usepackage{adjustbox}
\usepackage{caption}
\usepackage{booktabs} % Top and bottom rules for tables
\usepackage{graphicx}
\usepackage{bbding}
\usepackage{pifont}
\usepackage{wasysym}
\usepackage{amssymb}
\newcommand{\xmark}{\text{\ding{55}}}
\usepackage{colortbl}

%------------------------------------------------------------
%   Headline Colours
%------------------------------------------------------------

%------------------------------------------------------------

%----------------------------------------------------------------------------------------
%	TITLE SECTION 
%----------------------------------------------------------------------------------------

\title{%
  \texorpdfstring{%
    \makebox[\linewidth]{%
      \makebox[0pt][l]{%
        \raisebox{\dimexpr-\height+\baselineskip}[0pt][0pt]
          {\includegraphics[height=2\baselineskip]{figs/fsu-seal-gold.png}}% Left logo
      }\hfill
      \makebox[0pt]{Analyzing Hardware Parameters in GPU based HPC Platform}%
      \hfill\makebox[0pt][r]{%
        \raisebox{\dimexpr-\height+\baselineskip}[0pt][0pt]
          {\includegraphics[height=2\baselineskip]{figs/fsu-seal-gold.png}}% Right logo
      }%
    }%
  }
  {Simulating Hardware Design Parameters for Future HPC Platform}}

\author{Saptarshi Bhowmik\textsuperscript{1}, Nikhil Jain\textsuperscript{2},, Abhinav Bhatele\textsuperscript{3} and Xin Yuan\textsuperscript{1}} % Author(s)

\institute{1. Florida State University, 2. NVIDIA, Inc, 3. University of Maryland} % Institution(s)

%----------------------------------------------------------------------------------------

\begin{document}

\addtobeamertemplate{block end}{}{\vspace*{2ex}} % White space under blocks
\addtobeamertemplate{block alerted end}{}{\vspace*{2ex}} % White space under highlighted (alert) blocks

\setlength{\belowcaptionskip}{2ex} % White space under figures
\setlength\belowdisplayshortskip{2ex} % White space under equations

\begin{frame}[t] % The whole poster is enclosed in one beamer frame

\begin{columns}[t] % The whole poster consists of three major columns, the second of which is split into two columns twice - the [t] option aligns each column's content to the top

\begin{column}{\sepwid}\end{column} % Empty spacer column

\begin{column}{\onecolwid} % The first column

%----------------------------------------------------------------------------------------
%	Goals
%----------------------------------------------------------------------------------------

\begin{alertblock}{Goals}
\begin{itemize}
\item \textbf{Use discrete-event simulations to study various hardware design parameters and their impact on the performance of HPC workloads}
\item \textbf{Study several parameters including nework bandwidth, number of GPUs per node  in the context of two most popular network topologies Fat-Tree and  1d-Dragonfly}
\end{itemize}
\end{alertblock}

%----------------------------------------------------------------------------------------
%	INTRODUCTION
%----------------------------------------------------------------------------------------
\vspace{-1em}
\begin{block}{Introduction}

Today's GPU networks are largely dependant on communication performance of applications over the network. \textsuperscript{[1]}. As such the design and environment effects of performance on application is very pronounced and needs to be studied in general.

\end{block}
%----------------------------------------------------------------------------------------
%	METHODS
%----------------------------------------------------------------------------------------
\vspace{-1em}
\begin{block}{Method}
\textbf{Define the desired network} \newline We use two network topology that is implemented in TraceR-CODES simulator, 1d - Dragonfly and Fat-Tree with Adaptive Routing.

\newline
\textbf{We use the Number of GPU's per node from 1-GPU/Node, 2-GPU/Node, 4-GPU/Node and 8-GPU/Node}
%\begin{itemize}
%\vspace{-1em}

\newline
\textbf{We also use different Link bandwidths which is the ratio of base bandwidth X (10 Gbps), from x/8, x/4, x/2, x, 2x, 4x, 8x.}
%\begin{itemize}
%\vspace{-1em}

%\item The new network model is created in CODES which takes the DOT file as input.
%\item Runs simulation on that network topology.
%\item Read an tapered fat tree which is similar to quartz.
%\item Exact quartz topology from the dot file.

%\end{itemize}
%\begin{figure}
%\centering
%\begin{minipage}{.45\textwidth}
%\begin{table}
%\begin{adjustbox}{width=\columnwidth,left}

%\begin{tabular}{|c|c|} \hline
%\hline
%\textbf{Parameters} & \textbf{Values}  \\ \hline
%\cellcolor{ProcessBlue!10}Number of Levels  & \cellcolor{ProcessBlue!10}3   \\ %   \hline
%Number of Terminals & 2688  \\    \hline
%\cellcolor{ProcessBlue!10}Number of Leaf Switches & %\cellcolor{ProcessBlue!10}84  \\    \hline
%Number of Aggregate Switches & 84  \\    \hline
%\cellcolor{ProcessBlue!10}Number of Core Switches & %\cellcolor{ProcessBlue!10}32  \\    \hline
%Number Of Pods & 11  \\    \hline
%\end{tabular}
%\end{adjustbox}


%\caption{Quartz Topology}
%\end{table}

%\end{minipage}
%\begin{minipage}{.45\textwidth}
%\centering
%\includegraphics[width=1\linewidth,height=0.5\linewidth]{qpods.pdf}
%\captionsetup{labelformat=empty}
%\caption{Quartz Pod,  8 leaf switch an 8 aggregate switch}
%\end{minipage}
%\end{figure}
\vspace{2em}
\begin{table}
\begin{adjustbox}{width=\columnwidth,left}

\begin{tabular}{|c|c|c|} \hline
\hline
\textbf{Traces} & \textbf{Computation} & \textbf{Communication} \\ \hline
\cellcolor{ProcessBlue!10}Stencil4d & \cellcolor{ProcessBlue!10}\xmark & \cellcolor{ProcessBlue!10}\CheckmarkBold  \\    \hline
Kripke & \CheckmarkBold & \xmark  \\    \hline
\cellcolor{ProcessBlue!10}Laghos & \cellcolor{ProcessBlue!10}\CheckmarkBold & \cellcolor{ProcessBlue!10}\xmark  \\    \hline
Subcomm-a2a & \xmark & \CheckmarkBold  \\    \hline
\cellcolor{ProcessBlue!10}Sw4lite & \cellcolor{ProcessBlue!10}\CheckmarkBold  & \cellcolor{ProcessBlue!10}\CheckmarkBold  \\    \hline
Amg & \CheckmarkBold  & \CheckmarkBold  \\    \hline
\end{tabular}
\end{adjustbox}


\caption{Application Traces}
\end{table}



\end{block}

%------------------------------------------------





%----------------------------------------------------------------------------------------

\end{column} % End of the first column

\begin{column}{\sepwid}\end{column} % Empty spacer column

\begin{column}{\twocolwid} % Begin a column which is two columns wide (column 2)

\begin{columns}[t,totalwidth=\twocolwid] % Split up the two columns wide column

\begin{column}{\onecolwid}\vspace{-.6in} % The first column within column 2 (column 2.1)
\vspace{-1em}
%----------------------------------------------------------------------------------------
%	Experimental Setup
%----------------------------------------------------------------------------------------

\begin{block}{Experimental Setup}

We are running 20 Workloads of randomly selected jobs from the above application from ranks 32, 64, 128, 256 and 512. We make sure that each application rank appears atleast 4 times across all workloads. 
\begin{table}
\begin{adjustbox}{width=\columnwidth,height=0.3\columnwidth,left,valign=m}

\begin{tabular}{|c|c|c|c|c|c|} \hline
%\hline
% \multicolumn{}{c}{Workload 1}\\\hline
\textbf{Jobs} & \textbf{32-Ranks} & \textbf{64-Ranks} & \textbf{128-Ranks}  & \textbf{256-Ranks} & \textbf{512-Ranks}\\ \hline
\cellcolor{ProcessBlue!10}\textbf{Kripke} & \cellcolor{ProcessBlue!10}\textbf{4} & \cellcolor{ProcessBlue!10}\textbf{4} & \cellcolor{ProcessBlue!10}\textbf{4} & \cellcolor{ProcessBlue!10}\textbf{4} & \cellcolor{ProcessBlue!10}\textbf{5}  \\ \hline

\textbf{Laghos} & \textbf{6} & \textbf{4} & \textbf{4} & \textbf{4} & \textbf{4}  \\ \hline

\cellcolor{ProcessBlue!10}\textbf{Subcomm-A2A} & \cellcolor{ProcessBlue!10}\textbf{4} & \cellcolor{ProcessBlue!10}\textbf{6} & \cellcolor{ProcessBlue!10}\textbf{7} & \cellcolor{ProcessBlue!10}\textbf{10} & \cellcolor{ProcessBlue!10}\textbf{4}  \\ \hline

\textbf{Stencil4D} & \textbf{4} & \textbf{4} & \textbf{8} & \textbf{8} & \textbf{4}  \\ \hline

\cellcolor{ProcessBlue!10}\textbf{Sw4lite} & \cellcolor{ProcessBlue!10}\textbf{8}  & \cellcolor{ProcessBlue!10}\textbf{5}  & \cellcolor{ProcessBlue!10}\textbf{4}  & \cellcolor{ProcessBlue!10}\textbf{8}  & \cellcolor{ProcessBlue!10}\textbf{5}  \\ \hline

\textbf{Amg} & \textbf{8} & \textbf{5} & \textbf{5} & \textbf{5} & \textbf{12}  \\ \hline

%\multicolumn{1}{l}{}\\

\end{tabular}

%\begin{tabular}{|c|c|} \hline
%\hline
%\multicolumn{2}{c}{Workload 2}\\\hline
%\textbf{Jobs} & \textbf{Ranks}  \\ \hline
%\cellcolor{ProcessBlue!10}\textbf{Subcomm-A2A} & \cellcolor{ProcessBlue!10}\textbf{64}  \\ \hline
%\textbf{Subcomm-A2A} & \textbf{64}  \\ \hline
%\cellcolor{ProcessBlue!10}\textbf{Subcomm-A2A} & \cellcolor{ProcessBlue!10}\textbf{128}  \\ \hline
%\textbf{Subcomm-A2A} & \textbf{512}  \\ \hline
%\cellcolor{ProcessBlue!10}\textbf{Laghos} & \cellcolor{ProcessBlue!10}\textbf{64}  \\ \hline
%\textbf{Laghos} & \textbf{64}  \\ \hline
%\cellcolor{ProcessBlue!10}\textbf{Kripke} & \cellcolor{ProcessBlue!10}\textbf{256}  \\ \hline
%\textbf{Laghos} & \textbf{512}  \\ \hline
%\cellcolor{ProcessBlue!10}\textbf{Kripke} & \cellcolor{ProcessBlue!10}\textbf{512}  \\ \hline
%\textbf{Subcomm-A2A} & \textbf{128}  \\ \hline
%\cellcolor{ProcessBlue!10}\textbf{Stencil4D} & \cellcolor{ProcessBlue!10}\textbf{32}  \\ \hline
%\multicolumn{1}{l}{}\\
%\multicolumn{1}{l}{}\\
%\end{tabular}
%\begin{tabular}{|c|c|} \hline
%\hline
%\multicolumn{2}{c}{Workload 3}\\\hline
%\textbf{Jobs} & \textbf{Ranks}  \\ \hline
%\cellcolor{ProcessBlue!10}\textbf{Laghos} & \cellcolor{ProcessBlue!10}\textbf{256}  \\ \hline
%\textbf{Subcomm-A2A} & \textbf{256}  \\ \hline
%\cellcolor{ProcessBlue!10}\textbf{Kripke} & \cellcolor{ProcessBlue!10}\textbf{512}  \\ \hline
%\textbf{Laghos} & \textbf{64}  \\ \hline
%\cellcolor{ProcessBlue!10}\textbf{Stencil4D} & \cellcolor{ProcessBlue!10}\textbf{512}  \\ \hline
%\textbf{Stencil4D} & \textbf{128}  \\ \hline
%\cellcolor{ProcessBlue!10}\textbf{Laghos} & \cellcolor{ProcessBlue!10}\textbf{32}  \\ \hline
%\textbf{Kripke} & \textbf{256}  \\ \hline
%\cellcolor{ProcessBlue!10}\textbf{Stencil4D} & \cellcolor{ProcessBlue!10}\textbf{256}  \\ \hline
%\textbf{Stencil4D} &\textbf{32}  \\ \hline
%\cellcolor{ProcessBlue!10}\textbf{Kripke} & \cellcolor{ProcessBlue!10}\textbf{64}  \\ \hline
%\textbf{Kripke} & \textbf{32}  \\ \hline
%\cellcolor{ProcessBlue!10}\textbf{Laghos} & \cellcolor{ProcessBlue!10}\textbf{64}  \\ \hline
%\end{tabular}
\end{adjustbox}
\caption{Number of occurrences of each Application per rank across all workloads}
\end{table}

%\begin{table}
%\begin{adjustbox}{width=\columnwidth,height=0.3\columnwidth,left}

%\begin{tabular}{|c|c|} \hline
%\hline
%\multicolumn{2}{c}{Workload 1}\\\hline
%\textbf{Jobs} & \textbf{Ranks}  \\ \hline
%\cellcolor{ProcessBlue!10}\textbf{Laghos} & \cellcolor{ProcessBlue!10}\textbf{32}  \\ \hline
%\textbf{Stencil4D} & \textbf{32}  \\ \hline
%\cellcolor{ProcessBlue!10}\textbf{Subcomm-A2A} & \cellcolor{ProcessBlue!10}\textbf{64}  \\ \hline
%\textbf{Laghos} & \textbf{512}  \\ \hline
%\cellcolor{ProcessBlue!10}\textbf{Laghos} & \cellcolor{ProcessBlue!10}\textbf{256}  \\ \hline
%\textbf{Subcomm-A2A} & \textbf{512}  \\ \hline
%\cellcolor{ProcessBlue!10}\textbf{Stencil4D} & \cellcolor{ProcessBlue!10}\textbf{256}  \\ \hline
%\textbf{Stencil4D} & \textbf{32}  \\ \hline
%\cellcolor{ProcessBlue!10}\textbf{Stencil4D} & \cellcolor{ProcessBlue!10}\textbf{128}  \\ \hline
%\textbf{Stencil4D} & \textbf{256}  \\ \hline
%\cellcolor{ProcessBlue!10}\textbf{Laghos} & \cellcolor{ProcessBlue!10}\textbf{64}  \\ \hline
%\textbf{Kripke} & \textbf{64}  \\ \hline
%\cellcolor{ProcessBlue!10}\textbf{Stencil4D} & \cellcolor{ProcessBlue!10}\textbf{256}  \\ \hline
%\textbf{Stencil4D} & \textbf{32}  \\ \hline

%\end{tabular}

% \begin{tabular}{|c|c|} \hline
% \hline
% \multicolumn{2}{c}{Workload 2}\\\hline
% \textbf{Jobs} & \textbf{Ranks}  \\ \hline
% \cellcolor{ProcessBlue!10}\textbf{Kripke} & \cellcolor{ProcessBlue!10}\textbf{32}  \\ \hline
% \textbf{Laghos} &\textbf{512}  \\ \hline
% \cellcolor{ProcessBlue!10}\textbf{Subcomm-A2A} & \cellcolor{ProcessBlue!10}\textbf{32}  \\ \hline
% \textbf{Stencil4D} & \textbf{128}  \\ \hline
% \cellcolor{ProcessBlue!10}\textbf{Stencil4D} & \cellcolor{ProcessBlue!10}\textbf{256}  \\ \hline
% \textbf{Subcomm-A2A} &\textbf{256}  \\ \hline
% \cellcolor{ProcessBlue!10}\textbf{Subcomm-A2A} & \cellcolor{ProcessBlue!10}\textbf{64}  \\ \hline
% \textbf{Stencil4D} & \textbf{32}  \\ \hline
% \cellcolor{ProcessBlue!10}\textbf{Laghos} & \cellcolor{ProcessBlue!10}\textbf{128}  \\ \hline
% \textbf{Stencil4D} & \textbf{256}  \\ \hline
% \cellcolor{ProcessBlue!10}\textbf{Stencil4D} & \cellcolor{ProcessBlue!10}\textbf{64}  \\ \hline
% \textbf{Laghos} & \textbf{32}  \\ \hline
% \cellcolor{ProcessBlue!10}\textbf{Stencil4D} & \cellcolor{ProcessBlue!10}\textbf{32}  \\ \hline
% \multicolumn{1}{l}{}\\

% \end{tabular}
% \begin{tabular}{|c|c|} \hline
% \hline
% \multicolumn{2}{c}{Workload 3}\\\hline
% \textbf{Jobs} & \textbf{Ranks}  \\ \hline
% \cellcolor{ProcessBlue!10}\textbf{Stencil4D} & \cellcolor{ProcessBlue!10}\textbf{32}  \\ \hline
% \textbf{Subcomm-A2A} & \textbf{512}  \\ \hline
% \cellcolor{ProcessBlue!10}\textbf{Laghos} & \cellcolor{ProcessBlue!10}\textbf{256}  \\ \hline
% \textbf{Kripke} & \textbf{512}  \\ \hline
% \cellcolor{ProcessBlue!10}\textbf{Stencil4D} & \cellcolor{ProcessBlue!10}\textbf{64}  \\ \hline
% \textbf{Kripke} & \textbf{256}  \\ \hline
% \cellcolor{ProcessBlue!10}\textbf{Subcomm-A2A} & \cellcolor{ProcessBlue!10}\textbf{32}  \\ \hline
% \textbf{Stencil4D} & \textbf{32}  \\ \hline
% \multicolumn{1}{l}{}\\
% \multicolumn{1}{l}{}\\
% \multicolumn{1}{l}{}\\
% \multicolumn{1}{l}{}\\
% \multicolumn{1}{l}{}\\
% \multicolumn{1}{l}{}\\

% \end{tabular}
% \end{adjustbox}
% \caption{Workload with 8 Tasks per Node}
% \end{table}
\end{block}

%----------------------------------------------------------------------------------------

\end{column} % End of column 2.1

\begin{column}{\onecolwid}\vspace{-.6in} % The second column within column 2 (column 2.2)

%----------------------------------------------------------------------------------------
%	Results 2
%----------------------------------------------------------------------------------------
\vspace{-1em}
\begin{block}{Results}
%$\bullet$ We see that without scaling the Stencil and Subcom jobs are behaving well, with error percentage less than 20 percent, although the Kripke jobs show most variations.
%\vspace{\baselineskip}
\newline
\textbf{Second}, Impact of Bandwidth and different GPU's per node on Application Performance.

\begin{figure}
\centering
\begin{minipage}{1\textwidth}
\centering
\includegraphics[width=1\linewidth, height=20cm]{figs/ftree-bw-mapping-comm-stencil.eps}
\captionsetup{labelformat=empty}
\caption{Ratio of Bandwidth,GPUs per node mapping for Stencil in Fat-Tree}
\end{minipage}
\end{figure}

\end{block}

%----------------------------------------------------------------------------------------

\end{column} % End of column 2.2

\end{columns} % End of the split of column 2 - any content after this will now take up 2 columns width


\begin{columns}[t,totalwidth=\twocolwid] % Split up the two columns wide column again

\begin{column}{\onecolwid} % The first column within column 2 (column 2.1)
\vspace{-1em}
%----------------------------------------------------------------------------------------
%	RESULTS 1
%----------------------------------------------------------------------------------------
\vspace{-2.5em}
\begin{block}{Results}
\textbf{First}, Impact of Number of GPU's per node on Application performance. 
\begin{figure}
\centering
\begin{minipage}{.45\textwidth}
\centering
\includegraphics[width=1\linewidth]{figs/dfly-x-mapping-comm.eps}
\captionsetup{labelformat=empty}
\caption{Communication Apps}
\label{fig:13a}
\end{minipage}
\begin{minipage}{.45\textwidth}
\centering
\includegraphics[width=1\linewidth]{figs/dfly-x-mapping-comp-new.eps}
\captionsetup{labelformat=empty}
\caption{Amg, Sw4lite Proxy}
\label{fig:13b}
\end{minipage}
\captionsetup{labelformat=empty}
\caption{1d-Dragonfly Topology}
\end{figure}

\begin{figure}
\centering
\begin{minipage}{.45\textwidth}
\centering
\includegraphics[width=1\linewidth]{figs/ftree-x-mapping-comm.eps}
\captionsetup{labelformat=empty}
\caption{Communication Apps}
\label{fig:13a}
\end{minipage}
\begin{minipage}{.45\textwidth}
\centering
\includegraphics[width=1\linewidth]{figs/ftree-x-mapping-comp-new.eps}
\captionsetup{labelformat=empty}
\caption{Amg, Sw4lite Proxy}
\label{fig:13b}
\end{minipage}
\captionsetup{labelformat=empty}
\caption{Fat-Tree Topology}
\end{figure}




\end{block}

%----------------------------------------------------------------------------------------

\end{column} % End of column 2.1

\begin{column}{\onecolwid} % The second column within column 2 (column 2.2)
\vspace{-1em}
%----------------------------------------------------------------------------------------
%	RESULTS 
%----------------------------------------------------------------------------------------



\begin{figure}
\centering
\begin{minipage}{1\textwidth}
\centering
\includegraphics[width=1\linewidth, height=18cm]{figs/dfly-bw-mapping-comm-subcom.eps}
\captionsetup{labelformat=empty}
\caption{Ratio of Bandwidth,GPUs per node for Subcom3d-a2a in 1d-Dragonfly}
\end{minipage}
\end{figure}
\begin{figure}
\centering
\begin{minipage}{1\textwidth}
\centering
\includegraphics[width=1\linewidth, height=18cm]{figs/ftree-bw-mapping-comp-sw4lite.eps}
\captionsetup{labelformat=empty}
\caption{Ratio of Bandwidth,GPUs per node mapping for Sw4lite in Fat-Tree}
\end{minipage}
\end{figure}

%\begin{figure}
%\centering
%\begin{minipage}{.45\textwidth}
%\centering
%\includegraphics[width=1\linewidth]{full11.pdf}
%\captionsetup{labelformat=empty}
%\caption{Tapered Fat Tree Topology}
%\label{fig:13a}
%\end{minipage}
%\begin{minipage}{.45\textwidth}
%\centering
%\includegraphics[width=1\linewidth]{net11.pdf}
%\captionsetup{labelformat=empty}
%\caption{Quartz Topology}
%\label{fig:13b}
%\end{minipage}
%\captionsetup{labelformat=empty}
%\caption{Predicted Vs Observed time for Workload-1 1-Task/Node}
%\end{figure}

%\begin{figure}
%\begin{minipage}{.45\textwidth}
%\centering
%\includegraphics[width=1\linewidth]{full18.pdf}
%\captionsetup{labelformat=empty}
%\caption{Tapered Fat Tree Topology}
%\label{fig:13a}
%\end{minipage}
%\begin{minipage}{.45\textwidth}
%\centering
%\includegraphics[width=1\linewidth]{net18.pdf}
%\captionsetup{labelformat=empty}
%\caption{Quartz Topology}
%\label{fig:13b}
%\end{minipage}
%\captionsetup{labelformat=empty}
%\caption{Predicted Vs Observed time for Workload-1 8-Task/Node}
%\end{figure}





%----------------------------------------------------------------------------------------

\end{column} % End of column 2.2

\end{columns} % End of the split of column 2

\end{column} % End of the second column

\begin{column}{\sepwid}\end{column} % Empty spacer column

\begin{column}{\onecolwid} % The third column

%----------------------------------------------------------------------------------------
%	% CONCLUSION
%----------------------------------------------------------------------------------------

\begin{block}{Conclusion}
\begin{itemize}
\item \textbf{GPUs per node} The communication intensive applications slowdown when the number of GPUs per node is increased, among the proxy applications only AMG shows slowdown.
\vspace{\baselineskip}

\item \textbf{Link Bandwidth} As we increase the number of GPU's per node. More bandwidth is needed to make the performance at par with lower GPU's per node configuration.
\vspace{\baselineskip}
\item In Subcom3d-a2a applications, applications with fewer ranks are performing better as more GPU's are mapped to one node, as there is more intra-node communication.
\end{itemize}
\end{block}



%----------------------------------------------------------------------------------------
%	FUTURE WORKS
%----------------------------------------------------------------------------------------

\begin{block}{Future Work}
\begin{itemize}
\item Every application has a sweet spot where the performance is the best, figure out the sweet spot for other HPC applications.
\vspace{\baselineskip}
\vspace{\baselineskip}
\item Try to study how other simulation environment and hardware design, such as NIC scheduling policies effect the performance of applications
\end{itemize}
\end{block}

%----------------------------------------------------------------------------------------
%	REFERENCES
%----------------------------------------------------------------------------------------

\begin{block}{References}

\nocite{*} % Insert publications even if they are not cited in the poster
\small{\bibliographystyle{unsrt}
\bibliography{sample}\vspace{0.55in}}
\vspace{-0.5em}
\small{\rmfamily{This work was performed under the auspices of the U.S. Department of Energy by Lawrence Livermore National Laboratory under Contract DE-AC52-07NA27344 (LLNL-POST-783752).}} 
\end{block}

%----------------------------------------------------------------------------------------

\end{column} % End of the third column

\end{columns} % End of all the columns in the poster

\end{frame} % End of the enclosing frame

\end{document}
